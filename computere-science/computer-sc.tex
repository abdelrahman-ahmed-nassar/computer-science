- 
    
    🌷كلية حاسبات ومعلومات
    لو تعرف حد فى كلية حاسبات ومعلومات يبق ى لازم تساعده بالبوست دا
    🌷لطلبة سنه أولى في جميع كليات الحاسبات والمعلومات  و اللي لسه تايهين مش عارفين يذاكروا منين
    
    🌷شرح المواد pdf
    🥀 Discrete mathematics [http://bit.ly/3blZOY7](http://bit.ly/3blZOY7)
    🥀 English for computer science [http://bit.ly/3bmVl7J](http://bit.ly/3bmVl7J)
    🥀 fundamentals of programming [http://bit.ly/38uUY9f](http://bit.ly/38uUY9f)
    🥀 fundamentas of computer science [http://bit.ly/39dSsU5](http://bit.ly/39dSsU5)
    🥀 physics [http://bit.ly/2L7FLCh](http://bit.ly/2L7FLCh)
    🥀حقوق انسان [http://bit.ly/3s6imlc](http://bit.ly/3s6imlc)
    🥀 Math [http://bit.ly/3nymZRw](http://bit.ly/3nymZRw)
    🥀 physcis  [http://bit.ly/3hXMAC9](http://bit.ly/3hXMAC9)
    🥀 CS [http://bit.ly/38rohcQ](http://bit.ly/38rohcQ)
    🥀 Electronics [http://bit.ly/35pFIbO](http://bit.ly/35pFIbO)
    🥀 English  [http://bit.ly/2MJ76ed](http://bit.ly/2MJ76ed)
    🥀 IS [http://bit.ly/35m3oOc](http://bit.ly/35m3oOc)
    
    🌷اهم القنوات على " يوتيوب " لشرح المواد
    
    🥀 البرمجة ( حسونة )
    [https://bit.ly/37R5TtJ](https://bit.ly/37R5TtJ)
    🥀 رياضة منفصلة
    [http://bit.ly/2WQYXGk](http://bit.ly/2WQYXGk)
    
    🥀 علوم حاسب ( number system )
    [http://bit.ly/37RLVPy](http://bit.ly/37RLVPy)
    
    🥀 قناه Khan Academy لشرح منهج math
    [http://bit.ly/2LyDNdO](http://bit.ly/2LyDNdO)
    🥀  قناه anaHr لشرح منهج math
    [http://bit.ly/388l14z](http://bit.ly/388l14z)
    🥀 قناه Pen&Paper لشرح منهج math
    [http://bit.ly/2LBaJlT](http://bit.ly/2LBaJlT)
    🥀  قناه محمد الدميني لشرح منهج math
    [http://bit.ly/3gVwsRo](http://bit.ly/3gVwsRo)
    🥀 قناه Online - FCIH  لشرح منهج physcis
    [http://bit.ly/2IXT6fk](http://bit.ly/2IXT6fk)
    🥀 قناه Michel van Biezen  لشرح منهج physcis
    [http://bit.ly/3aeUWDA](http://bit.ly/3aeUWDA)
    🥀 قناه CS50  لشرح منهج  cs
    [http://bit.ly/3gTdudW](http://bit.ly/3gTdudW)[http://bit.ly/3gQf00x](http://bit.ly/3gQf00x)[http://bit.ly/2Whdlrr](http://bit.ly/2Whdlrr)[http://bit.ly/2KxKTyB](http://bit.ly/2KxKTyB)[http://bit.ly/3nr1nHo](http://bit.ly/3nr1nHo)
    🥀 قناه MUST TEAM  لشرح منهج  Electronics
    [http://bit.ly/2KvK7m1](http://bit.ly/2KvK7m1)
    🥀 قناه MA TECH  لشرح منهج  Electronics
    [http://bit.ly/3p0jEvw](http://bit.ly/3p0jEvw)[http://bit.ly/3nvaB5O](http://bit.ly/3nvaB5O)[http://bit.ly/37odwHF](http://bit.ly/37odwHF)
    🥀 قناه   لشرح منهج  English
    [http://bit.ly/2K1qanj](http://bit.ly/2K1qanj)
    🥀 ازى تبدا فى مجال البرمجه من الصفر حتى الاحتراف ( خطه + الماتريال  والفيديوهات اللى هتحتاجها
    
    [http://bit.ly/3r4buo3](http://bit.ly/3r4buo3)
    
    🌷ملفات pdf شرح وتلخيصات
    
    🥀 ملخص حقوق الانسان ( جامعة المنصورة )
    [http://bit.ly/2JlTpAE](http://bit.ly/2JlTpAE)
    🥀شرح CS
    [http://bit.ly/2KZ4aJw](http://bit.ly/2KZ4aJw)
    🥀 شرح Electronics
    [http://bit.ly/3aJTuJK](http://bit.ly/3aJTuJK)
    🥀شرح English
    [http://bit.ly/2L2Sbek](http://bit.ly/2L2Sbek)
    🥀شرح IS[http://bit.ly/37NNu0K](http://bit.ly/37NNu0K)
    🥀 شرح Math
    [http://bit.ly/3mPaG2R](http://bit.ly/3mPaG2R)
    🥀 شرح physcis
    [http://bit.ly/3rxAswg](http://bit.ly/3rxAswg)
    🥀 ملخص ال css
    [http://bit.ly/2McBgqb](http://bit.ly/2McBgqb)
    🥀 ملخص ال html
    [http://bit.ly/2McBgqb](http://bit.ly/2McBgqb)
    🥀 ملخص ال css3
    [http://bit.ly/38B9dIo](http://bit.ly/38B9dIo)
    🥀ملخص ال html5
    [http://bit.ly/3rDVg5p](http://bit.ly/3rDVg5p)
    🥀كتاب الكترونكس
    [http://bit.ly/3hxqPsZ](http://bit.ly/3hxqPsZ)
    🥀كتاب فيزكس
    [http://bit.ly/38Ay4fz](http://bit.ly/38Ay4fz)
    🥀 قاموس مصطلحات برمجه
    [http://bit.ly/34OxM3t](http://bit.ly/34OxM3t)
    
    🥀محاضرات بايثون
    [https://bit.ly/Python-Courses](https://bit.ly/Python-Courses)
    🥀محاضرات في شبكات الكمبيوتر
    [https://bit.ly/34oWCXT](https://bit.ly/34oWCXT)
    🥀محاضرات جافا
    [https://bit.ly/31X3Gtp](https://bit.ly/31X3Gtp)
    🥀محاضرات في هندسة البرمجيات
    [https://bit.ly/34t8pV8](https://bit.ly/34t8pV8)
    🥀محاضرات ++C
    [http://bit.ly/C-Plus-Plus-Courses](http://bit.ly/C-Plus-Plus-Courses)
    🥀كورس سي شارب من الصفر حتى الاحتراف
    [https://bit.ly/3dUEt7v](https://bit.ly/3dUEt7v)
    🥀كورسات في أساسيات البرمجة
    [https://bit.ly/Programming-Courses-101](https://bit.ly/Programming-Courses-101)
    🥀محاضرات في لغة C
    [https://bit.ly/3mjLvFH](https://bit.ly/3mjLvFH)
    🥀كورسات في التشفير
    [http://bit.ly/Cryptography-Courses](http://bit.ly/Cryptography-Courses)
    🥀محاضرات Algorithm باللغة العربية
    [https://bit.ly/35ANajL](https://bit.ly/35ANajL)
    🥀كورسات Cybersecurity
    [http://bit.ly/Cybersecurity-Courses-101](http://bit.ly/Cybersecurity-Courses-101)
    🥀محاضرات تفاضل وتكامل
    [http://bit.ly/Calculus-Courses](http://bit.ly/Calculus-Courses)
    🥀محاضرات IT باللغة العربية
    [https://bit.ly/37GOVyy](https://bit.ly/37GOVyy)
    🥀محاضرات ذكاء اصطناعي
    [http://bit.ly/AI-Courses](http://bit.ly/AI-Courses)
    🥀كورسات في الروبوتات
    [http://bit.ly/Robot-Courses](http://bit.ly/Robot-Courses)
    🥀محاضرات Data Structures بالعربي
    [https://bit.ly/2G0OaEY](https://bit.ly/2G0OaEY)
    🥀محاضرات جبر
    [http://bit.ly/ِAlgebra-Courses](http://bit.ly/%D9%90Algebra-Courses)
    🥀محاضرات PHP باللغة العربية
    [https://bit.ly/34rg1HA](https://bit.ly/34rg1HA)
    🥀قنوات يوتيوب في تعليم البرمجة
    [http://bit.ly/Programming-YouTube-Channels](http://bit.ly/Programming-YouTube-Channels)
    🥀كورس Data Structures كامل
    [https://bit.ly/3dVkC82](https://bit.ly/3dVkC82)
    🥀محاضرات في لغة التجميع (Assembly Language)
    [https://bit.ly/3oqkZfS](https://bit.ly/3oqkZfS)
    
    🥀 افضل موقع هتلاقوا عليه كورسات وبرامج كتير:
    [https://bit.ly/36aunvu](https://bit.ly/36aunvu)
    🥀كورسات اونلاين مجانية في مجال أمن المعلومات
    [https://bit.ly/31N3tZD](https://bit.ly/31N3tZD)
    
    لو تعرف حد فى حاسبات ومعلومات او مهتم بمجال البرمجه ساعدة بالبوست دا
    ياريت الناس تتابع البيدج هتفيدكم جدا
    سواء كنت مبرمج محترف أو مبتدئ، إمسك البوست ده مينفعش يعدي منك 😲لو كنت محترف برمجة أو لسه باديء أو حتى لسه بتتعلم، فـ انت كده كده هتحتاج مجموعة من الأكواد الجاهزة اللي تختصرلك الوقت، أو اللي من خلالها تقدر تتعلم أداء مجموعة من المهمات اللي هتحتاجها في طريقك للاحتراف، وكمان هتحتاجها في مشاريعك 👨‍💻
    هنا بقى هقولك على أفضل مواقع هتلاقي فيها أكواد جاهزة مفتوحة المصدر، وهتفيدك جدًا: 👇😉 Ashraf404
    
    1️⃣️ موقع Github 👇
    موقع Github من المواقع الضخمة والكبيرة جدًا واللي بتجمع كتير من المبرمجين في مختلف مناطق العالم، وموقع قديم جدًا، وبيتيح للمبرمجين والمطورين مشاركة مشاريعهم الخاصة بمصدر مفتوح مع باقي الناس أو مع الناس اللي بتتابعهم على حسابهم، وموجود في الموقع آلاف المطورين والمبرمجين، وبيسمحلك تأخد المشاريع اللي إنت عايزها، والأكواد الجاهزة للتعديل، وكمان محرر عشان تحرير الأكواد البرمجية من الموقع وتجربتها.
    وده لينك الموقع : [https://github.com/](https://github.com/)
    لو عايز تعرف ازاي تتعامل مع الموقع شوف الفيديو ده 😉 [https://youtu.be/1WA21wIW4pI](https://youtu.be/1WA21wIW4pI)
    
    2️⃣️ موقع searchcode 👇
    ده بقى من أهم المواقع بالنسبة لأعظم المبرمجين، لأنه بيحتوي على أكتر من 20 بليون سطر برمجي، وأكتر من 7 ملايين مشروع برمجي، وده أكتر من اللي انت ممكن تحتاجه أصلًا، وكمان الموقع مجاني فتقدر تأخد المشروع اللي إنت عايزه أو الكود اللي إنت عايزه براحتك.
    وده لينك الموقع : [https://searchcode.com/](https://searchcode.com/)
    
    3️⃣️ موقع Open Hub 👇
    موقع ضخم جدًا، هتلاقي فيه كل اللي يخص الأكواد البرمجية، وأكتر من 30 بليون سطر برمجي مخصص ليك، وكمان يعتبر محرك بحث على الأكواد.
    ده لينك الموقع : [https://www.openhub.net/](https://www.openhub.net/)
    
    4️⃣️ موقع CodeProject 👇
    لو بقى عايز مشاريع كاملة في كل اللغات البرمجية يبقى هتلاقي طلبك في الموقع ده، ده موقع تابع لشركة مايكروسوفت خاص بالمبرمجين وفيه الآلاف من المطورين والمبرمجين من جميع أنحاء العالم، وكلهم بيشاركوا أعمالهم و مشاريعهم على شكل ملفات مفتوحة المصدر، وتقدر تنسخ وتعدل عليها وكمان تطرح أسئلتك بكل سهولة.
    ده لينك الموقع : [https://www.codeproject.com/](https://www.codeproject.com/)
    
    5️⃣️ موقع Codota 👇
    لو انت مطور تطبيقات أندرويد فده أفضل موقع ممكن تلاقيه، وتقدر تأخد أكواد التطبيقات وكل حاجة تحتاجها تخص تطبيقات الأندرويد من عليه، والموقع فيه طريقة إحترافية للبحث عن أكواد مفتوحة المصدر.
    ده لينك الموقع : [https://www.codota.com/](https://www.codota.com/)
    
    6️⃣️ موقع [CodePen.io](http://codepen.io/) 👇
    لو انت بقى مصمم أو مبرمج ويب، الموقع ده هيفيدك جدًا، ده بقى موقع تواصل إجتماعي للمبرمجين ومصممي مواقع الويب بيسمحلك الموقع بكتابة الأكواد البرمجية وإدراتها وكمان مشاركتها مع غيرك، وكمان الإستفادة من تصميمات غيرك. والموقع كمان بيحتوي على محرر أكواد، وكمان تقدر تحصل منه على تعليقات وتقييم لمشروعك على الموقع.
    ده لينك الموقع : [https://codepen.io/](https://codepen.io/)
    
    7️⃣️ موقع CodeAnyWhere 👇
    الموقع ده زي موقع CodePen بالظبط، بس ده نسخة مطورة وإحترافية وبيتميز بمجموعة من المميزات كتدعيمه للغات اللي بتتعامل مع السيرفرات زي php و python و java . والموقع ده بيحتوى على أكتر من 350 مليون ملف من الأكواد مفتوحة المصدر، وبيسمحلك الحصول على الأكواد بكل سهولة، وكمان مشاركتها بطريقة إحترافية.
    ده لينك الموقع : [https://codeanywhere.com](https://codeanywhere.com/)
    
    7️⃣موقع stack overflow 👇
    لو انت مبرمج اومصمم فالموقع ده هيفيدك جدا ده بقى تسأل عن اى مشكله في كل اللغات البرمجية يبقى هتلاقي طلبك في  الموقع ده
    ده لينك الموقع : [https://stackoverflow.com/](https://stackoverflow.com/)
    
    بالتوفيق للجميع <3